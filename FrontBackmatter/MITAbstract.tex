%*******************************************************
% Titlepage
%*******************************************************
\begin{titlepage}
    % if you want the titlepage to be centered, uncomment and fine-tune the line below (KOMA classes environment)
    %\begin{addmargin}[-1cm]{-3cm}
      {\setlength{\parindent}{0cm}
        \large  


        \hfill


        \begingroup
            \hyphenpenalty=10000        

            \color{Maroon}\spacedallcaps{\myTitle} \\ 
            \mySubtitle \\ 

            \bigskip
        \endgroup

        \spacedlowsmallcaps{\myName}\\ \medskip

   Submitted to the Program in Media Arts and Sciences, School of Architecture and Planning, on \myTime in partial fulfillment of the requirements for the degree of Master of Science in Media Arts and Sciences at the Massachusetts Institute of Technology. \\ 

\bigskip
\spacedallcaps{Abstract}\\ \medskip

Collaborations between Makers and Nonprofits are unique. They give makers an opportunity to practice their skills, understand their value and have a real world impact. In turn, Nonprofits gain different perspectives on their challenges and get to collaborate towards solutions that otherwise may not be in their reach. These collaborations also pose unique challenges. Nonprofits with fewer resources will often have difficulties defining their problems, not to mention specifying a solution. They do, however, have a compelling story to tell, the story of their mission, their processes and their constraints. Makers, with their unique cross-discipline skill set can extract challenges, problems and solutions from these stories. This thesis outlines this process of story-centric brainstorming and collaboration, and presents a web application, \textit{This is How}, in which stories are represented as videos, enriched with timeline based discussion and collaborative pads.  
\vfill


Thesis Supervisor:\\
Professor Andrew Lippman\\
Associate Professor of Media Arts and Sciences\\
Program in Media Arts and Sciences

        }
  %\end{addmargin}       
\end{titlepage}   
