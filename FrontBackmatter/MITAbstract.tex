%*******************************************************
% Titlepage
%*******************************************************
\begin{titlepage}
    % if you want the titlepage to be centered, uncomment and fine-tune the line below (KOMA classes environment)
    %\begin{addmargin}[-1cm]{-3cm}
      {\setlength{\parindent}{0cm}
        \large  


        \hfill


        \begingroup
            \hyphenpenalty=10000        

            \color{Maroon}\spacedallcaps{\myTitle} \\ 
            \mySubtitle \\ 

            \bigskip
        \endgroup

        \spacedlowsmallcaps{\myName}\\ \medskip

   Submitted to the Program in Media Arts and Sciences, School of Architecture and Planning, on \myTime in partial fulfillment of the requirements for the degree of Master of Science in Media Arts and Sciences at the Massachusetts Institute of Technology. \\ 

\bigskip
\spacedallcaps{Abstract}\\ \medskip

Almost two hundred years into the industrial revolution, the ability to develop dedicated
machines has increased the efficiency of businesses around the world. However, for many small businesses the concept of custom machinery for their own unique requirements is still foreign. At the same time, the maker movement is seeing explosive growth. There are currently thousands of active makerspaces in the world and that number increases on a month to month basis. These makerspaces facilitate creative minds and allow the creation of internet connected devices, 3d printed prosthetics and much more. How can we harness the power of the maker revolution to help small community businesses? One of the issues in doing so is that these businesses are not aware of the full breadth of engineering possibilities available and do not know how to articulate their needs. However, they can often tell a compelling story about their operation. I propose a web platform, This is How, in which small businesses share their stories in the form of video bytes in which they explain what they do and why, what are their requirements and constraints and what kind of issues they have. Makers can then annotate the video, ask further questions and propose solutions for issues.
\vfill

Thesis Supervisor:\\
Professor Andrew Lippman\\
Associate Professor of Media Arts and Sciences\\
Program in Media Arts and Sciences

        }
  %\end{addmargin}       
\end{titlepage}   
