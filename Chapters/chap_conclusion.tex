\chapter{Conclusion}
\label{chap_conclusion}

The goal of this thesis was to explore new ways to foster collaboration between the maker community and nonprofit organizations through face-to-face interaction and web-based interaction on the \textit{This Is How} platform. 

First, an outline of the process of story­ centric brainstorming and collaboration was developed. Through examination of case studies and interviewing the parties involved I demonstrated that these type of collaborations contribute to the stated goals. Makers gained confidence in their skills and their ability to make an impact, and nonprofits gained different perspectives on the challenges they face, collaborated towards solutions and learned about the maker movement. 

Second, I designed and implemented a web application, \textit{This is How}, that distributes the outlined process by using HyperMedia as a method for interactive exploration and collaborative notepads as basis for collaboration. By interviewing the same nonprofits and makers from previous examples regarding \textit{This is How} I established that it has the potential to replicate the process of story­ centric brainstorming and collaboration in a distributed manner.     

\section{Limitations}

\subsection{Storytelling through video is not easy}

The stories in \textit{This is How} are based on videos which need to strike a balance between telling a compelling story, having enough details to attract makers and yet refrain from being cumbersome. While the three nonprofits covered in this thesis have stated that they believe creating such a video is within their capacity, that is yet to proven. I think the guidelines presented in the previous chapter are a good starting point and I'm encouraged by how well these nonprofits are in presenting their story in person. I believe that with an initial corpus of successful videos in the system, nonprofits will be able to utilize imitation as a tool for creating a compelling video story.

\subsection{Variance in the Maker movement}

Most makers interviewed in this thesis are from the MIT network, they are extremely capable and socially aware. However, The very definition of a Maker is undefined, Makers range from expert engineers in various fields to kids in elementary school practicing arts and crafts. More than that, the motivation for engaging in ``making'' is also varied, from personal expression to social activism and more. 

Having visited various maker spaces in the greater Boston area I believe that the best candidates for \textit{This is How} are makers that are already organized in frameworks with a tendency for social good, such as the previously mentioned \textit{Learn to Teach, Teach to Learn} program. To jump start \textit{This is How} it is crucial to connect and collaborate with these organizations.

\subsection{Variance in Nonprofits}

The three nonprofits covered in this thesis differ greatly from one another but still represent an extremely small sample size. Nonprofits vary in size, characteristics of operation and numerous other factors. Many of them focus solely on the distribution of money and have no use for the maker skill set and some bigger nonprofits, such as \textit{Cradles to Crayons}, work at a scale in which it is harder to experiment and have other considerations such as public relations.
It is clear that for a nonprofit to benefit from a platform such as \textit{This is How} they should be small enough to allow experimentation and have a ground operation that benefits from the maker skill set. As the system grows, the characterization of an ideal nonprofit candidate will be refined.   

\subsection{Single Iteration} 

This is How in it's current form is a first iteration. In the world of software development, especially software that's provided as a service like web applications, it is necessary to observe users' behavior and constantly refine and validate both the underlying assumptions and the specifics of user experience.

\section{Future work}

The next main step for \textit{This is How} is deployment. Evaluating the system with a small set of nonprofits and makers has been invaluable for the development process but can not be considered a predictor for success.

Deployment requires more than just opening the system up for the general public, it requires building and maintaining a community. This entails actively seeking relevant nonprofits and helping them tell their story in a concise way that can allure makers. 

Deployment also means to approach maker communities, both virtual and physical, and promote the system. Hopefully, with a critical mass of stories and ongoing collaborations, the existing users, both makers and nonprofits, will become ambassadors of the system and attract their peers.

I'm looking forward to proceed with the development of \textit{This is How} and am excited about the potential impact of deployment.  

\section{Reflections}

Working on this thesis has had a tremendous impact on me.

Collaborating with nonprofits has taught me many lessons about humility and selflessness. My trips to various nonprofits and the meetings with their employees and volunteers kept reminding me of the importance of my work but more than that, they served as a much needed moral compass for a student spending most of his time in the over-achieving environment of MIT.

I was also inspired by my visits to maker spaces, outside of MIT, where the sparse resources led to creative thinking. I was reminded what it means to learn and to build out of curiosity rather than out of a requirement to innovate. Getting a positive response from the kids of the South End Technology Center was probably the most exciting moment in this entire process.

With that said, there were also many days and weeks spent on directions that turned out to be dead ends. At times, I was frustrated by lack of success in enlisting makers and nonprofits, to the point of questioning the basic assumptions of this thesis.

Still, I became convinced that these type of collaborations are extremely beneficial to all parties involved. Whether \textit{This is How} is the right tool to facilitate them is still an open question, which I wish to further explore, but I believe it's a step in the right direction. 

Moving forward I can safely say that regardless of the future of this project, it has had a tremendous impact on me and will stay with me wherever I  go.  