% introduction
%  - vision/overview
%     - a decade from now robots will take the place along side humans and animals
%     - as companions, playmates, tutors, assistants
%     - there is a change in what we have to be, what we have to regard as peers to make this happen
%     - john henry vs robots
%  - why empathy for robots
%  - theory of life story
%  - description for rest of thesis

\chapter{Conclusion}
\label{chap_conclusion}

The goal of this thesis was to foster collaboration between the maker community and nonprofit organizations. This was done through two main contributions. The first, Outlining the process of story­centric Brainstorming and collaboration. Second, build a web platform, \textit{This is How} that enables the distribution of the same type of collaborations.

\section{Future work}

The next main step for This is How is deployment. Evaluating the system with a small set of nonprofits and makers has been invaluable for the development process but can not be considered a predictor for success.

Deployment requires more than just opening the system up for the general public, it requires building and maintaining a community. This entails actively seeking relevant nonprofits and helping them tell their story in a concise way that can allure makers. 

Deployment also means approaching maker communities, both virtual and physical, and promote the system. Hopefully, with a critical mass of stories and ongoing collaboration, the existing users, both makers and nonprofits, will be the ambassadors of the system and attract their peers.

\section{Reflections}

Working on this thesis has had a tremendous impact on me.

Collaborating with nonprofits has taught me many lessons about humility and selflessness. My trips to various nonprofits and the meetings with their employees and volunteers kept reminding me of the importance of my work but more than that, they served as a much needed moral compass for a student spending most of his time in the meritocracy of MIT.

I was also inspired by my visits to maker spaces, outside of MIT, where the sparse resources led to creative thinking. I was reminded what it means to learn and to build out of curiosity rather than out of a requirement to innovate. Getting a positive response from the kids of the South End Technology Center was probably the most exciting moment in this entire process.

With that said, there were also many days and weeks spent on directions that turned out to be dead ends. At times, I was frustrated by lack of success in enlisting makers and nonprofits, to the point of questioning the basic assumptions of this thesis.

Moving forward, I still can't say if I'm going to proceed with this project but I know for sure that it will stay with me wherever I go.