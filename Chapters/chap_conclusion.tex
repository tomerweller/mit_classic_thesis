\chapter{Conclusion}
\label{chap_conclusion}

The goal of this thesis was to foster collaboration between the maker community and nonprofit organizations. This was done through two main contributions. 

First, an outline of the process of story­ centric brainstorming and collaboration was presented. Through examination of case studies and interviewing the parties involved I demonstrated that these type of collaborations contribute to the stated goals. Makers gained confidence in their skills and their ability to make an impact, and nonprofits gained different perspectives on the challenges they face, collaborated towards solutions and learned about the maker movement. 

Second, I designed and implemented a web application, \textit{This is How}, that distributes the outlined process by using HyperMedia as a method for interactive exploration and collaborative idea documents as basis for collaboration. By interviewing the same nonprofits and makers from previous examples regarding \textit{This is How} I established that it has the potential to replicate the process of story­ centric brainstorming and collaboration in a distributed manner.     

\section{Limitations}

\subsection{Storytelling through video is not easy}

The stories in  This is How are based on videos which need to strike a balance between telling a compelling story, having enough details to attract makers and yet refrain from being cumbersome. While the three nonprofits covered in this thesis have stated that they believe creating such a video is within their capacity, that is yet to proven. I'm encouraged by how well these nonprofits are in presenting their story in person and believe that with an initial corpus of successful videos in the system, nonprofits will be able to utilize imitation as a tool for creating these video stories.

\subsection{Small Sample Size}. 

This thesis is based on the nature of three collaboration stories and evaluated on a qualitative basis. While the nonprofits and makers that have been interviewed come from a relatively wide range of background, the actual variance in these communities is huge. Nonprofits vary in size, character of their operation and countless other factors, while the very definition of Maker is undefined. Makers range from expert engineers in various fields to kids in elementary schools practicing arts and crafts. It is naive to believe that what works for these few cases will work for all. For wide scale deployment, There will have to be an ongoing process of refinement, which leads to the next section.

\subsection{Single Iteration} 

This is How in it's current form is a first iteration. In the world of software development, especially software that's provided as a service like web applications, it is necessary to observe users' behavior and constantly refine and validate both the underlying assumptions and the specifics of user experience.

\section{Future work}

The next main step for This is How is deployment. Evaluating the system with a small set of nonprofits and makers has been invaluable for the development process but can not be considered a predictor for success.

Deployment requires more than just opening the system up for the general public, it requires building and maintaining a community. This entails actively seeking relevant nonprofits and helping them tell their story in a concise way that can allure makers. 

Deployment also means to approach maker communities, both virtual and physical, and promote the system. Hopefully, with a critical mass of stories and ongoing collaborations, the existing users, both makers and nonprofits, will be the ambassadors of the system and attract their peers.

\section{Reflections}

Working on this thesis has had a tremendous impact on me.

Collaborating with nonprofits has taught me many lessons about humility and selflessness. My trips to various nonprofits and the meetings with their employees and volunteers kept reminding me of the importance of my work but more than that, they served as a much needed moral compass for a student spending most of his time in the meritocracy of MIT.

I was also inspired by my visits to maker spaces, outside of MIT, where the sparse resources led to creative thinking. I was reminded what it means to learn and to build out of curiosity rather than out of a requirement to innovate. Getting a positive response from the kids of the South End Technology Center was probably the most exciting moment in this entire process.

With that said, there were also many days and weeks spent on directions that turned out to be dead ends. At times, I was frustrated by lack of success in enlisting makers and nonprofits, to the point of questioning the basic assumptions of this thesis.

Moving forward, I still can't say if I'm going to proceed with this project but I know for sure that it will stay with me wherever I go.