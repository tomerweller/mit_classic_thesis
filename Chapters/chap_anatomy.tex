\chapter{Anatomy of Distributed Story-Centric Brainstorming and Collaboration }
\label{chap_anatomy}

This thesis proposes a system for story-centric distributed brainstorming and collaboration. story-centric collaborations between makers and nonprofits are collaborations that are a result of the stories told by nonprofits. These stories represent the goals, processes and constraints of the nonprofits. While these nonprofits will not always have the expertise to define a challenge a specification of a solution, they are well versed in their story.

Before designing a distributed system it is crucial to understand how these type of collaborations work in small scale. To that purpose I present three case studies of story centric brainstorming and collaboration. These case studies are on based testimonials from relevant organizations and participants. At the end of the Chapter I present a common process extracted from these cases. 

\section{Milky}
This case study represents my own experience in working with the Mothers' Milk Bank Northeast \cite{mmne}. This collaboration started in the summer of 2015, before my thesis research started, and served as motivation for the research direction I chose.

The Mothers' Milk Bank Northeast is a non profit community organization that receives donations of human milk, processes them and provides them to babies in fragile health throughout the Northeastern United States. Breast milk is considered to be the healthiest source of nutrients for babies and especially important in cases of preterm birth, infectious disease and various other medical conditions.\cite{} However, for various reasons, sometimes the birth mother can not lactate. The milk bank collects donated milk and makes it available through hospitals. 

I learned about the milk bank completely by chance. My peer, Tal Achituv, had co-organized a hackathon which was dedicated to the improvement of breast milk pumps \cite{d2016feminist}. One of the attendees, Yavni Bar-Yam, was the son of Naomi Bar-Yam, who runs the milk bank. Yavni told Tal about the organization, who in turn convinced me to come for a visit. I had just finished my first year as a grad student at the media lab and had an itch for industrial design and hardware fabrication.

In the visit, we learned about the process breast milk donations go through from collection until being received in hospitals. This included processing of the milk: mixing it, pasteurizing and bottling. One of the things we noticed was that the tools used by the lab workers were very basic. The milk was hand poured through a series of vessels and finally into the bottles, where the main measurement tool was eyeballing the volume markings. 

% Figure: Manual processing 

This manual process became the focal point of our discussion. While it was cumbersome it also presented issues of accuracy and being prone to infections. The milk bank had tried to address these issues before using an off the shelf peristaltic pump instead of manually pouring. However, it slowed down the pace of work and did not allow for parallelization between lab workers. 

We suggested a simple solution for the problem, building a carousel bottle feeder in conjunction with a peristaltic pump. The carousel will allow continuous feeding of empty bottles on one side and capping and removing the bottles on the other side. We called this device Milky. 

% Figure: Milky, early prototype

We built several iterations of Milky during a period of several months while being in constant dialog with Milk Bank. We did so in collaboration with the organization's employees. The lab workers, who will eventually use the machine, gave us feedback early along the process to make sure the device fits in the lab and is suitable for their workflow. David, consulted us regarding materials, sanity and made sure that the process does not prove to be harmful for the milk. 

While Milky hasn't been deployed yet, given the delicate nature of the operation, the preliminary feedback is encouraging. Another achievement of the project, beyond the actual device, was that  the people of the milk bank became aware of the digital fabrication movement. Before this collaboration, they were convinced that fabricating custom machinery is only within the reach of large corporations such as Coca Cola. They realized, that they might be able to turn to local makerspaces with their problems and ideas. 

In developing Milky, I became convinced that there is an unrealized opportunity for makers and nonprofits to collaborate in a way that could be beneficial for both parties.


\section{Laser Painting}

Deborah Dawson, an artist from New York, helps kids with cerebral palsy to create art. She does so by attaching a small laser pointer to their heads and have them point at a painting canvas. In turn, she traces their pointer to create an abstract painting. This allows kids with very little ability to express themselves, to create art they can call their own.

% Figure : Tracing

However, Deb had a challenge with mounting these devices. Laser pointers are not designed to be worn on the head. This proves to be very difficult, especially by kids that suffer from cerebral palsy and have limited motor skills. Her quest to find a more suitable alternative led her to companies that manufacture lasers sights built for guns. Naturally, these companies could not help. Eventually, on a long shot, she sent an email describing her issues to an MIT Media Lab Prof., Ros Picard, who in turn sent it to all Media Lab students (Appendix). 

% figure Deb Dawson in session 

Tal Achituv, spotted that email and invited Deb to MIT. Deb, Tal and several other students, including myself, met with her. We learned about her work and suggested a headband design with an integrated low power laser pointer for the kids. Moreover, through our conversation we learned about her entire operation which led to a discussion over some other challenges she faces. For example, the fact the she worked in different schools and in different spaces meant that all the equipment needed to be carried around. Not to mention, locating spaces that had the accessibility requirements for these kids was also challenging. One of the ideas we discussed was repurposing an old school bus with wheelchair accessibility and the painting infrastructure. That way the equipment only needs to be setup once and the bus could travel around New York and cater to various schools in different locations. 

% figure of the band

The alternative headband laser pointer was delivered to Deb in the Spring of 2016 and has been in use since then. Moreover, Deb is currently looking to expand her operations and is in touch with maker spaces in New York to fabricate more of them. Also, Deb is actively seeking funding for the repurposing of a bus as a portable studio for her work. 

This collaboration has also inspired Tal to further explore the world of assisted self expression.  For his thesis project he built a painting machine that assists people with various disabilities to express themselves by drawing, with minimal assistance from other people\cite{projexpress}. 

% Project Express

From this collaboration I learned that while nonprofits will often seek help with a small, narrow challenge, there is an opportunity for both parties in discussing the bigger picture: the story. 

\section{Cradles to Crayons}

Following the previous two study cases, I set out to initiate a more structured exploration into the space. In contrast to our previous experiences, which happened completely by chance, this time I actively sought out a potential collaboration. 

The initial challenge was to find a suitable organization. Given that nonprofit organizations matched the profile, I approached the Massachusetts Nonprofit Network (MNN). The MNN agreed to publish a call for collaboration in their weekly newsletter which yielded some 50 requests from various nonprofits (appendix). In the requests, one stood out as having a complex ground operation that might be of interest to makers, Cradles to Crayons.   

``Cradles to Crayons (C2C) provides children from birth to age 12, living in low-income and homeless situations, with the essential items they need to thrive – at home, at school and at play. We supply these items free of charge by engaging and connecting communities that have with communities that need.'' 

<TODO: copied from website. paraphrase.>

The C2C operation is a complex logistic one. It starts with donation bins distributed in local communities and goes through sorting and filtering in their main facility, and ends in creating custom packs for children in need, which are distributed through local social workers. 

%  Figure C2C tour

We arranged for a party of 15 makers from different backgrounds to visit the C2C facility in Brighton, MA. Julia, director of community engagement for C2C led the tour. Julia received no instructions from us apart from showing around the facility the same way she would in any other circumstance. During the tour the participants learned about the different challenges the organization faces: from the security of their donations bins to difficulties of managing stock. This learning process was backed by a continuous discussion between the participants, Julia and other employees of the organization. 

At the end of the tour the participants gathered in an office to discuss. Given that C2C have a complex operation, the first thing they did is formalize the entire operation as a block diagram on a whiteboard, utilizing one of the participants as a moderator. Next, each participant was given a sticky note to write one idea on, and stick it on the whiteboard around the most relevant block. This quickly formed clusters of sticky notes around identified blocks and triggered a discussion about the different problems and the various proposed solution. 

% Figure Brainstorming

The proposed ideas varied in nature. One of them, a mobile application, tackled the inventory issues by allowing donors to create a manifest of their donations in advance. Another, suggested cerating a direct link, using a mobile app, between the donors and the social workers, thus reducing congestion in the facility. Others were as simple as modifying the graphics on the bins to better instruct the donors what belongs in them and what doesn't. 

While non of the visiting makers are currently collaborating with C2C, the idea of creating a direct link between donors and social workers through a mobile app has been approved by management and is in an initial phase of implementation. The congestion challenge this app addresses was not part of the challenges originally laid out by Julia but rather surfaced from the broad story of the organization, showing the power of these stories.  

\section{Impressions}

Examining the above cases clearly demonstrate that the concept of story-centric brainstorming and collaboration works towards the goals I stated in the introduction. All the nonprofits mentioned have stated that they have gained a valuable different perspective on the challenges they're facing while learning about the abilities and skillsets of the makers they've worked with. In turn, the makers learned about the different operations of these nonprofits and how their skillsets can help improve them. 

These goals were achieved regardless of the outcome of the collaboration. Even in cases where the collaboration has not produced a successful artifact (yet), both parties have agreed that the process itself was beneficial.

\section{Formalizing the process}

Based on these three case studies I present a common process that has three phases. While these cases focus on nonprofit organizations and participants with background in engineering, there is still some variance in the different challenges and nature of collaboration. I believe this process is generalized enough to be used to tackle various challenges with different organizations and participants. 

<TODO add some evaluation from interviews>

\begin{enumerate}

\item \textit{Discovery} - Discovery is perhaps the most elusive step of this process but is a basic requirement for it's existence. In the two first cases discovery happened completely by chance while the third one was the result of an intended search. How do organizations learn about the existence of these type of participants and vice versa?

\item \textit{Exploration} - In all of the above cases, participants had at least one on-site visit with the respective organizations. These explorations started from a general overview and ended in a detailed drill down into the processes employed by the organization. Open discussion during this exploration proves to be a major role in understanding the organization: motivation, processes and constraints. 

\item \textit{Collaborative Brainstorming} - The next step in all of these examples is a process of collaborative brainstorming. This included tossing around many ideas and, filtering and refining them according to the given constraints. These sometimes began in a spontaneous meeting right after the exploration phase but also went on to remote forms of communications be it phone, email etc.

\end{enumerate}

 
Having established that these type of collaborations work towards the goals I stated, the next challenge is to translate the outlined process into an online experience that can be scaled while preserving the unique properties of the real life experience. 