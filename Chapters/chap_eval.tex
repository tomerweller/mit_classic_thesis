\chapter{Evaluation}
\label{chap_eval}

I've conducted structured interviews (Appendix) with all three nonprofits mentioned in Chapter 3 as well as three makers who participated in some sort of collaboration with them. These nonprofits vary in scale and in type of operation.

\section{User Experience Evaluation}

The user experience evaluation centered around a demonstration given by my self. The demonstration started from the main story page, including commenting and idea pads, and then moved on to browsing and story creation. 


Per step, the interviewee was requested to rank the ease of use, according to their personal experience. They were also given an option to elaborate. I present a summary of each of the steps.

\subsection{Video}

\subsection{Story Comments System}

The subjects were shown an existing video story with comments, and comment replies, both in the form of text and video.

There was a unanimous agreement between subjects that once an existing comment in the video appears, the timeline-based discussion is self explanatory. Some of the interviewees stated that clicking on a comment to open the reply-to-comment system is not intuitive, However, once having seen it, the options to reply by text or video are self explanatory and valuable.

Makers and nonprofits alike have stated that this step seems to be true to the purpose of interactive exploration, as experienced in their collaborations. Some subjects thought the asyncronous nature of the online discussion has more potential than real life interactions because it gives the parties a chance to think about their replies. Others stated the opposite, that the asyncronous nature removes a sense of urgency which exists in real-life discussion and drives it. 

\subsection{Story Creation}

This step focused on the nonprofits, being the story creators, which all stated that the form and mechanics of uploading a file were straightforward and self explanatory, including the usage of tags as keywords. 

This step also served as a trigger for discussion regarding the feasibility of producing a compelling video story. The two smaller nonprofits stated that they believe it's within their reach to produce such a video while C2C, the largest and most established one, raised a concern regarding video quality. All media released to the public domain by them must be vetted to comply with the organization's public relation's team which has high standards.   

\subsection{Idea Pad}

The subjects were shown an Idea Pad with existing ideas regarding the same story as the previous section. Some of these ideas have already been collaborated on and discussed on by various users.

All the nonprofits stated that it is hard to evaluate without real usage although it seems to be straightforward and fits the process of brainstorming. C2C, the larger nonprofit and also the most aware of public relations, was worried that non-makers who visit the page, will not understand the goal of the idea pad.   

Two of the makers appreciated the lack of structure in the pad as a tool for free form brainstorming while the third was worried of the exact opposite, that the lack of structure does not accommodate the relationships between ideas.

\subsection{Browsing}

The subjects were shown the main page along with the option to filter by tags.

Nonprofits were neutral in their response while makers expressed a concern about scalability, how the page well look like with many stories, including the lack of a free form search option.

\subsection{Overall User Experience}

Makers and nonprofits alike have stated that in general the user interface provides a concise and self explanatory experience. The novel components, including the commenting system and the idea pad, became clear once they saw them populated with data.     


\section{"This is How" as a platform}
Subjects were asked whether they believe 